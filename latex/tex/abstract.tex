\keywords{multi-view stereo, reconstruction, 3D scanning, computer graphics, imaging}
\begin{abstractpage}[english]
Rendering of photorealistic models is becoming increasingly feasible and important in computer graphics.
Due to the high amount of work in creating such models by hand, required by the need of high level of detail in geometry, colors, and animation, it is desirable to automate the model creation.
This task is realised by recovering content from photographs that describe the modeled subject from multiple viewpoints.
In this thesis, elementary theory on image acquisition and photograph-based reconstruction of geometry and colors is studied and recent research and software for graphics content reconstruction are reviewed.
Based on the studied background, a rig is built as a grid of nine off-the-shelf digital cameras, a custom remote shutter trigger, and supporting software.
The purpose of the rig is to capture high-quality photographs and video in a reliable and practical manner, to be processed in multi-view reconstruction programs.

The camera rig was configured to shoot small static subjects for experiments.
The resulting photos and video were then processed directly for the subject's geometry and color, with little or no image enhancements done.
%The rig was shown to work well for the purpose of multi-view reconstruction by photographing several subjects.
Two typical reconstruction software pipelines were described and tested.
The rig was found to perform well for several subjects with little subject-specific tuning;
issues in the setup were analyzed and further improvements suggested based on the captured test models.
Image quality of the cameras was found to be excellent for the task, and most problems arose from uneven lighting and practical issues.
The developed rig was found to produce sub-millimeter scale accuracy in geometry and texture of subjects such as human faces.
Future work was suggested for lighting, video synchronization and study of state-of-the-art image processing and reconstruction algorithms.
\end{abstractpage}

\newpage

\keywords{monen näkymän stereo, rekonstruktio, 3D-skannaus, tietokonegra-fiikka, kuvantaminen}
\begin{abstractpage}[finnish]
Fotorealististen mallien renderöinti on yhä tärkeämpää ja mahdollisempaa tieto-konegrafiikassa.
Näiden mallien luominen käsityönä on työlästä vaaditun korkean tarkkuuden takia geometrian, värien ja animaation osalta, ja onkin haluttavaa kor-vata käsityö automatiikalla.
Automatisointi voidaan suorittaa taltioimalla sisältö valokuvista, jotka kuvastavat samaa kohdetta useammasta näkymästä.
Tässä diplomityössä tarkastellaan kuvantamista ja valokuvapohjaisen geometrian rekonstruktiota geometrian ja värien kannalta ja katsastetaan viimeaikainen tutkimus ja ohjelmistot grafiikkasisällön rekonstruointiin.
Taustan pohjalta rakennetaan laitteisto, joka koostuu yhdeksästä valmiina saatavilla olevasta digikamerasta aseteltuna hilaksi, itse tehdystä etälaukaisimesta sekä ohjelmistoista.
Laitteiston tarkoituksena on taltioida luotettavalla ja käytännöllisellä tavalla korkealaatuisia valokuvia ja videokuvaa, joita voi käsitellä monen näkymän stereon rekonstruktion tietokonesovelluksissa.

Laitteisto säädettiin kuvaamaan pieniä staattisia kohteita kokeita varten.
Tuloksena saadusta kuvamateriaalista laskettiin kohteen geometria ja värit ilman mainittavaa kuvanparannuksen käyttöä.
Käytiin läpi kaksi tyypillistä rekonstruktio-ohjelmistoa testiksi ja laitteiston havaittiin soveltuvan hyvin useisiin kohteisiin ilman erityistä säätämistä.
Kameroiden kuvanlaatu todettiin tehtävään erinomaiseksi ja useimmat haasteet johtuivat epätasaisesta valaistuksesta ja käytännön pulmista.
Laitteiston todettiin tuottavan alle millimetriskaalan geo-metriaa ja pintavärikuvaa ihmiskasvojen kaltaisista kohteista.
Jatkotyötä ehdotettiin valaistukseen, videon synkronointiin ja viimeisimpien kuvankäsittely- ja re-konstruktioalgoritmien tutkimiseen.
\end{abstractpage}
