\keywords{multi-view stereo, reconstruction, 3D scanning, computer graphics, imaging}
\begin{abstractpage}[english]
Although we live in a three-dimensional world, flat two-dimensional images of it are common.
Computer graphics synthesizes visual content to a two-dimensional screen based on a given three-dimensional geometric model.
Content is completely described as geometry, colors, and lighting models, but modeling accurate, realistic geometry by hand is time-consuming and difficult.

In this thesis, the necessary background on image acquisition and principles of multi-view stereo reconstruction are studied.
Based on the theoretical background, easily available cameras for a reconstruction rig are selected, and a rig is built with accompanying software for easy and practical image acquisition for reconstructing static and moving subjects.
Using off-the-shelf commercial cameras only, a model of a real-life scene with geometry and color is reconstructed from photographs taken from multiple viewpoints in the scene, a process known as stereo reconstruction or 3D scanning.

Multi-view stereo reconstruction acts as a confluence between computer graphics and computer vision.
Computer vision studies the inverse process: reconstructing a three-dimensional model from its two-dimensional projections in photographs.
Quality of the photographed imagery is directly related to the accuracy of the reconstructed model in both geometric precision and color texture resolution.
Present state of reconstruction software is surveyed in this thesis.
Several test subjects are scanned with the rig and reconstruction results of selected software packages are presented.
The developed rig is shown to produce sub-millimeter scale accuracy in geometry and texture of subjects such as human faces.
Finally, conclusions on the system feasibility and proposals for future research are given.
\end{abstractpage}

\newpage

\keywords{}
\begin{abstractpage}[finnish]
	TODO translate when ready
\end{abstractpage}
