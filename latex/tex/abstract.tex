\keywords{3D, multi-view stereo, reconstruction, computer graphics, imaging}
\begin{abstractpage}[english]
Multi-view stereo reconstruction acts as a confluence between computer graphics and computer vision.
Computer graphics synthesizes visual content to a two-dimensional screen based on a three-dimensional model.
Computer vision studies the inverse process: reconstructing a three-dimensional model from its two-dimensional projections in photographs.
Using off-the-shelf commercial cameras only, a model of a real-life scene with geometry and color can be reconstructed from photographs taken from multiple viewpoints in the scene, a process known as stereo reconstruction or 3D scanning.
Quality of the photographed imagery is directly related to the accuracy of the reconstructed model in both geometric precision and color texture resolution.

In this thesis, the necessary background on image acquisition and principles of multi-view stereo reconstruction are studied.
Based on the theoretical background, cameras for a reconstruction rig are selected, and a rig is built with accompanying software for easy and quick image acquisition for reconstructing static and moving subjects.
Present state of reconstruction software is surveyed.
Several test subjects are scanned and reconstruction results of selected software packages are presented.
The developed rig is shown to produce sub-millimeter scale accuracy in geometry and texture of subjects such as human faces.
Finally, conclusions on the system feasibility and proposals for future research are given.
\end{abstractpage}

\newpage

\keywords{}
\begin{abstractpage}[finnish]
\end{abstractpage}
