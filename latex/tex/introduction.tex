% Johdanto
\section{Introduction}

% no page numbers and stuff
\thispagestyle{empty}

\subsection{Background and motivation}

% introintro

The real world where we live in consists of three-dimensional (3D) objects.
Photographs and drawings of it, on the other hand, are flat, only two-dimensional (2D).
A photograph is shot with a camera by making a two-dimensional \emph{projection} of the light traveling in the world on a flat film.

% introduce computer graphics and the need for content

In \emph{computer graphics}, a great interest is the digital synthesis of visual content.
A geometric 3D model of a scene is given, and the light transport in it is synthesized to produce a flat projection from a single view.
3D computer graphics has numerous applications in video games, motion pictures, medical sciences, architectural engineering, archaeology, and more.
%Video games with animated, authentic faces of real persons are probably some of the most familiar applications to the general public.
%Precise measurements based on reconstructed 3D models are increasingly important landscape mapping and in 3D printed body parts.
Among most applications, content of high photorealistic detail is of interest.
The increasing processing power of computer hardware makes it possible to meet this requirement.
Constructing realistic three-dimensional structures by hand is difficult and time-consuming, though.
Scanning objects and scenes in real life for automatically building 3D content is becoming increasingly popular.

% what is 3d content, technically; connection to computer vision

The structure of 3D content is universally described as surfaces consisting of connected polygons, mapped with pictures to add high-detailed color.
While computer graphics synthetizes images by drawing millions of polygons to produce a realistic image, \emph{computer vision} tries to attain the inverse:
to study images computationally for understanding the visual information.
Among the applications of computer vision, reconstructing a 3D structure from two-dimensional images is an important one.
3D reconstruction naturally unites computer vision and computer graphics.

% bring mvs in

%Computer vision is a large and mature field; the area for acquiring 3D structure from pictures is well known, and the increasing computing power enables the methods to evolve and to even move to mobile devices, making multi-view stereo reconstruction a hot topic.

\emph{Multi-view stereo (MVS)} is a method for extracting the 3D geometry and color information of a subject, given several images of it taken from different angles (\emph{views}).
MVS can be applied to an unordered collection of pictures or to the imagery of a more strict set of calibrated cameras.
A MVS scanning rig is a specially constructed machinery, consisting of a number of cameras and software, for shooting photos from different angles for computer vision processing.

% mvs compared to plain stereo

Plain \emph{stereo vision} based on fixed two-camera views has been in use in many applications varying from the robotics industry to 3D movie recording.
Stereo vision compares the views of two cameras for depth cues, using similar principles as the human binocular vision.
Multi-view stereo incorporates a multitude of cameras to acquire a more wide field of view at the same time.
The process of acquiring multi-view stereo images and processing them for original 3D structure is called \emph{3D scanning}, \emph{3D capture} or \emph{3D reconstruction}.

% static/dynamic definition, surface vs. motion capture, texture

The need for \emph{dynamic} content divides 3D scanning in two parts.
\emph{Static} subjects, i.e.\ stationary rigid bodies, present a simpler problem.
Dynamic (time-varying) subjects move or deform over time, such as human bodies, skin, and clothing.
Static capture is simpler to accomplish, as the time when the pictures are taken does not affect the outcome; even a single moving camera can be used.

%Dynamic capture has long been used in special effects for movies, for capturing a human body performance with a large set of static cameras, encoding the positions of separate limbs over time.

%(Other applications (uses). Movies. Remedy. Medical [essential physics of, bushberg]!).
%Landscape/architecture engineering. Crime scenes (police investigation). Topographic / terrain mapping. Geology, archaeology. Object replication with 3d printing. Aerial photography (digital elevation models DEM)

% on motion capture

Human \emph{motion capture} is one application of dynamic 3D reconstruction.
Traditional ways to do motion capture use a set of reflective markers attached on a special suit to recover only the motion of limbs over time, often encoded as parameters of skeletal joint angles.
The motion information is then reinterpreted for a 3D model to move its parts virtually.
\emph{Surface capture} refers to a more detailed scan of the motion of a complete surface, such as a human face or a cloth, possibly augmented on a skeletal motion capture data for a complete body.
Capturing the movement in much higher resolution to also scan the whole surface movement and deformation is a more recent interest, requiring much more computational power.
Advances in camera resolution make it possible to use only surface texture fiducials without separate markers.

% starting our interests; humans in movies+games

%The Matrix \cite{wachowski99matrix}, a movie famous of its bullet-time scenes, used heavy optical flow processing to ``slow time down'';
%then, The Matrix Reloaded \cite{wachowski03reloaded} used a technique called Universal Capture by Borshukov et al. \cite{borshukov05universal} to capture the skin surface of a human head, in order to replicate the head in the form of computer models.
%Large scale static 3D scanning and coloring of archeologic sceneries can be done with combining laser scanning and photographs \cite{lerma2010terrestrial}.
%Outside scenes need special care \cite{vu2012high}.
%Small static objects have been scanned successfully with a turntable \cite{fitzgibbon1998automatic}, and structure from motion techniques have recently been presented that impressively recover a structure from pictures taken all over an outdoor location with no a priori information about the camera configurations. \cite{goesele2007multi,furukawa2010towards}.

Recent movies and video games (such as \cite{rockstar2011noire}) have shown that by capturing real motion of facial expressions instead of simulating them produces significant results.
%Literature on capture of human faces and skin surface motions is large. \cite{deng2007computer}
While tools exist for reconstructing simple objects using only a single consumer camera, professional quality capture still needs a laboratory environment with a large number of cameras \cite{winder2008technical,motionscan}.

% XXX less background, more contents and motivation

\subsection{Goal and scope of the thesis}

The objective of this thesis is to construct a complete scanning rig to be used as a part of an end-to-end pipeline for scanned 3D content generation.
The thesis is divided in two parts.
First, the relevant theoretical background on digital image acquisition and multi-view stereo reconstruction is surveyed.
After that, a more practical part follows: the constructed system is described and experiment scans for both shape and texture information are presented.

The research problem of ``scanning an object in 3D'' is formulated in such a way that proper hardware can be selected, and the quality and feasibility of the built system can be addressed.
Performance of the reconstruction algorithms depends on the quality of their input.
How an image is formed with a digital camera, what cameras are available in the market, and how a good input for a reconstruction algorithm can be achieved?
To further assess and understand the resulting quality of the reconstruction, the fundamentals of the reconstruction algorithms are presented.

A digital camera based 3D scanning rig is constructed, supported by software tools for acquiring both static image data and video files.
The key hardware for such rig consists of properly placed cameras, light sources, and a computer.
Putting together all the parts for a general-purpose system needs studying of the end product as a whole.
The camera synchronization and image acquisition are automated such that the output of the scanner can be fed directly to a reconstruction software without manual processing.
Final result produced by the rig is a set of images or video, ready to be processed by reconstruction algorithms for producing a 3D mesh with detailed colors, possibly animated over time.
Feasible readily available programs for reconstructing 3D geometry and texture based on the captured image data are surveyed and listed, and two popular workflows are tested.
Basic background techniques for taking also motion into account are presented, but the use of state-of-the-art algorithms is left for future implementations.

Aim of the system is in hardware robustness, ease of use and general extendability.
Because there are no detailed plans on all further use of the machinery, it is built to support many kinds of studies in the future without requiring the user to know all implementation issues in detail, and documenting the process in the form of this thesis.
A special case subject in this work is human face: an interesting target because of its complex surface material and diverse ability to produce many different expressions.

The thesis is organised in eight sections, of which this introduction is the first.
Section \ref{sec:image-acquisition} presents the theoretical background on the elementary methods in image acquisition using a digital camera and the issues in video recording.
Relevant 3D reconstruction theory is described in section \ref{sec:static3d}, with focus on the basics that recent state-of-the-art builds on.
Section \ref{sec:motioncapture} extends the 3D topic by introducing issues on motion capture and reconstructing dynamic subjects.
In section \ref{sec:implementation}, the implementation of the 3D scanning rig is described.
Sections \ref{sec:experiments} and \ref{sec:discussion} show test experiments on the rig and discuss their results.
Section \ref{sec:conclusion} proposes future work and concludes the thesis.
