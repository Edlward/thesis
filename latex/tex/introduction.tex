% Johdanto
\section{Introduction}

% no page numbers and stuff
\thispagestyle{empty}

\subsection{Background and motivation}

% introintro

The real world where we live in consists of visually three-dimensional (3D) objects.
Photographs and drawings of it, on the other hand, are flat, only two-dimensional (2D) \emph{projections}.
Specifically, a camera takes a projection of a scene by capturing the light traveling around in the scene, originated from light sources, ending up to the flat film or digital image sensor.
The result is a distorted view of the 3D dimensions and colors, depending on illumination and camera pose.

% introduce computer graphics and the need for content

In \emph{computer graphics}, a great interest is the digital \emph{synthesis} of visual 3D content \cite{foley1990computer}.
A geometric 3D model of a scene is given, and the light transport in it is synthesized to produce a flat projection from a single view, as a camera would show the scene.
3D computer graphics has numerous applications in video games, motion pictures, medical sciences, architectural engineering, archaeology, and more.
Among most applications, content of high photorealistic detail is of interest and increasingly needed.
The constantly increasing and already high level of processing power in computer hardware makes it possible to meet this requirement.
Constructing realistic three-dimensional structures by hand is difficult and time-consuming, though.
The goal of this thesis is to address this problem by using \emph{3D reconstruction}: current state of digital cameras enables to create content directly by scanning the shape and color of real 3D objects using standard digital photography and computer vision.
Scanning objects and scenes in real life for automatically building 3D content is becoming increasingly popular.
In this thesis, such scanning methods are studied and a rig for performing photographic 3D scans is built.

% what is 3d content, technically; connection to computer vision

The structure of 3D content is universally described as surfaces consisting of connected polygons, mapped with pictures to add high-detailed color.
While computer graphics synthesizes images by drawing millions of polygons to produce a realistic image, \emph{computer vision} tries to attain the inverse:
to study images of 3D surroundings computationally for understanding the geometric structure in them.
Among the applications of computer vision, reconstructing such 3D structure from two-dimensional images is an important one.
3D reconstruction naturally unites computer vision and computer graphics.

% bring mvs in

\emph{Multi-view stereo (MVS)} \cite{hartley03multiview} is a method for extracting the visible 3D geometry and color information of a subject, given several images of the subject taken from different angles (\emph{views}).
MVS can be applied to an unordered collection of pictures or to the imagery of a more strict set of calibrated cameras.
A MVS scanning rig is a specially constructed machinery, consisting of a number of cameras and software, for shooting photos from different views and processing the photos using computer vision.

% mvs compared to plain stereo

Plain \emph{stereo vision}, MVS based on fixed two-camera views, has been long used in many applications varying from the robotics industry to 3D movie recording.
Stereo vision compares the views of two cameras for depth cues, using similar principles as the human binocular vision.
Multi-view stereo incorporates a multitude of cameras to acquire a more comprehensive and accurate view at the same time.
The practice of acquiring multi-view stereo images and processing them for original 3D structure is called \emph{3D scanning}, \emph{visual 3D capture}, \emph{3D reconstruction} or \emph{photogrammetry}.

% static/dynamic definition, surface vs motion capture, texture

The need for \emph{dynamic} content divides this particular application of 3D scanning in two types.
\emph{Static} subjects, i.e., stationary rigid bodies, present a simpler problem.
Dynamic subjects, such as human limbs, skin, and clothing, move or deform over time.
Static capture is simpler to accomplish, as the time when the pictures are taken does not affect the outcome; even a single moving camera can be used.

%(Other applications (uses), Movies, Remedy, Medical [essential physics of, bushberg]!).
%Landscape/architecture engineering, Crime scenes (police investigation), Topographic / terrain mapping, Geology, archaeology, Object replication with 3d printing, Aerial photography (digital elevation models DEM)

% on motion capture

Human \emph{motion capture} is one application of dynamic 3D reconstruction, with a goal to capture whole-body motion.
Traditional ways to do motion capture use a set of bright markers attached on a special suit to recover only the motion of limbs over time, often further encoded as parameters of skeletal joint angles.
The motion information is then reinterpreted for a 3D model to move its parts virtually.
\emph{Surface capture} refers to a more detailed scan of the motion of a complete surface, such as a human face or a cloth, possibly augmented on a skeletal motion capture data for a complete body.
Capturing the movement in high resolution to scan the whole surface deformation and colors is a recent interest, requiring much more computational power than traditional motion capture.
Advances in camera resolution and speed make it possible to use only surface texture details for tracking motion without separate markers.
Recent movies and video games have shown that by capturing real motion of facial expressions instead of simulating them produces realistic high-quality results.

\subsection{Goal and scope of the thesis}

% quick intro on contents

The objective of this thesis is to construct a complete scanning rig to be used as a part of an end-to-end pipeline for scanned 3D content generation, using multi-view stereo methods.
The thesis is divided in two parts.
First, the relevant theoretical background on photographic digital image acquisition and multi-view stereo reconstruction is surveyed.
After that, a more practical part follows: the constructed system is described and test scans for both shape and texture information are presented.
It should be noted that this work deals with well-known theory only and the theory is approached as is relevant to the goal of this work.
Deeply studying state-of-the-art and developing new theory is only suggested for future work.

% what is the first part about; its purpose and motivation

The research problem of ``scanning an object visually in 3D'' is formulated in such a way that proper hardware can be selected, and the quality and feasibility of the built system can be addressed.
Performance of the reconstruction algorithms depends on the quality of their input.
How an image is formed with a digital camera, what cameras are available in the market, what are the typical issues in the reconstruction process, and how can a good 2D input for a 3D reconstruction algorithm be achieved?
To further assess and understand the resulting quality of the reconstruction, the fundamentals of the reconstruction algorithms are presented, and state-of-the-art MVS research is quickly surveyed for quality expectations.

% second part

A digital camera based 3D scanning rig is constructed, supported by software tools for acquiring both static image data and video files.
The key hardware for such a rig consists of properly placed cameras, light sources, and a computer.
Putting together all the parts for a general-purpose system needs studying of the end product as a whole.
The camera synchronization and image acquisition are automated such that the output of the scanner can be fed directly to a reconstruction software without manual processing.
The final result produced by the rig is a set of images or video, ready to be processed by reconstruction algorithms for producing 3D geometry with detailed colors, possibly animated over time.
Feasible readily available programs for reconstructing 3D geometry and texture given the captured image data are surveyed and listed, and two popular workflows are tested.
Basic background techniques for taking also motion into account are presented, but the use of state-of-the-art algorithms is left for future implementations.

% work objective, motivation specs

The aim of the system is in hardware robustness, ease of practical use and general extendability.
Because there are no detailed plans on all further use of the machinery, it is built to support many kinds of studies in the future without requiring the user to know all implementation issues in detail, and documenting the process in the form of this thesis.
A special case subject in this work is human face: an interesting and challenging target because of its complex surface material and diverse ability to produce many different expressions.

% organisation

The thesis is organised in seven sections, of which this introduction is the first.
Section \ref{sec:image-acquisition} presents the general theoretical background on the elementary methods in image acquisition using a digital camera and the issues in video recording.
Relevant 3D reconstruction theory is described in Section \ref{sec:static3d}, with focus on the basics that recent state-of-the-art builds on.
Section \ref{sec:motioncapture} briefly extends the 3D topic by introducing issues on motion capture and reconstructing dynamic subjects.
In Section \ref{sec:implementation}, the implementation of the 3D scanning rig is described.
Section \ref{sec:experiments} shows test experiments on the rig and discusses their results.
Section \ref{sec:conclusion} concludes the thesis with proposals on future work.
