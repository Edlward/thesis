% Johdanto
\section{Introduction}

%(wikipedia: In recent decades, there is an important demand for 3D content for computer graphics, virtual reality and communication, triggering a change in emphasis for the requirements. Many existing systems for constructing 3D models are built around specialized hardware (e.g. stereo rigs) resulting in a high cost, which cannot satisfy the requirement of its new applications. This gap stimulates the use of digital imaging facilities (like a camera). Moore's law also tells us that more work can be done in software. An early method was proposed by Tomasi and Kanade.[1] They used an affine factorization approach to extract 3D from images sequences. However, the assumption of orthographic projection is a significant limitation of this system.)

% Performance-Driven Facial Animation, Lance Williams, Computer Graphics, Volume 24, Number 4, August 1990
%%http://en.wikipedia.org/wiki/Facial_motion_capture

%% Leave first page empty
% (wat?)
\thispagestyle{empty}

\subsection{Background and motivation}

% small intro, what's scanning
Computer vision is a mature field; the steps of acquiring three-dimensional structure of a real-life scene are well known, and there are several methods available.

Multi-view stereo capture/scanning can be described as acquiring the geometry and color information of a subject from several images taken from different poses.
A stereo scanning rig is a specially constructed machinery, consisting of two or more cameras, for quickly taking photos from different viewpoints for machine vision processing.

The geometric structure of an object can be extracted from distinct features that can be detected with several cameras.

The popularity of handheld computers - smartphones - equipped with relatively good quality cameras is currently an attractive target for 3d scanning software, while the actual reconstruction works in the cloud.
Kinect, the active infrared-based scanner for the xbox console, has reached a phase where it can be considered ubiquitous in gaming.
Current advances in camera technology and price make it possible for individual consumers to do pretty sophisticated reconstruction that has been possible only with a large budget of e.g. a movie studio.
Even software that uses only a smart phone camera and runs on the same phone has been published.

% static/dynamic definition, surface vs. motion capture, texture

3D scanning can be divided roughly in two: static and dynamic subjects.
Static case considers a rigid and non-moving subject, while dynamic scanners are used for moving bodies or deforming surfaces such as human skin or clothing.
Static capture is simple in the way that the time when the pictures are taken does not affect the outcome; a single moving camera can be used.
Dynamic motion capture has long been used for capturing a human body performance with a large set of static cameras, encoding the positions of separate limbs.
Capturing also the moving texture (surface capture, as opposed to motion capture) while recording the geometry movement is a related, much more demanding task that has become more popular with the increase in computing power.

% applications

(Other applications (uses). Movies. Remedy. Medical [essential physics of, bushberg]!). Landscape/architecture engineering. Crime scenes (police investigation). Topographic / terrain mapping. Geology, archaeology. Object replication with 3d printing. Aerial photography (digital elevation models DEM)

Large scale static 3D scanning and texturing of archeologic sceneries is done with combining laser scanning and photographs \cite{lerma2010terrestrial}.

Small objects have been scanned successfully with a turntable \cite{fitzgibbon1998automatic}, and structure from motion techniques have recently been presented that impressively recover a structure from pictures taken all over an outdoor location with no a priori information about the camera configurations. \cite{goesele2007multi,furukawa2010towards}.

Static applications are more developed; while dynamic scanning has lots of interest in scanning faces and human motion, also static cases have received attention in medical applications, landscape/architecture engineering and mapping, and crime scene investigation. % are they actually?

% on motion capture

Three-dimensional motion capture means in general the way of recording a sequence of movements of a real-life target or scene.
The technical details vary by what information is required and available.
When time is considered as one additional dimension, sometimes 3D motion captured sequences are called 4D video.
This reflects e.g. the names of some software packages \cite{something4dgoogleitup}

Facial surface motion capture (``mocap'') is currently a standard tool in the movie and video game entertainment industry among the more mature mocap for whole body movement to record performances for replaying them later.
This capture records movements of a human body (or another object that is to be recorded) so that the movements can be replayed or analyzed. Traditional methods only model the joint angles and consider parts between them as rigid bodies, tracking tens of points in realtime. \cite{todo}
Level of detail varies, depending on what is needed; from simple bone movements to deformations of skin, muscles and clothing.
Reconstructing skin and cloth motion is another highly nonrigid application that can be scanned for acquisition of model parameters. \cite{pritchard2003cloth}

Human performance capture is traditionally done with special easily distinguishable markers, typically small retroreflective spheres, that are mapped to a polygonal model for playback.
Playback then interpolates between the recorded positions.
Advances in camera resolution make it possible to use only surface texture fiducials without separate markers.
% http://www.siggraph.org/education/materials/HyperGraph/animation/character_animation/motion_capture/history1.htm

% starting our interests; humans in movies+games

The Matrix \cite{wachowski99matrix}, a movie famous of its bullet-time scenes, used heavy optical flow processing to ``slow time down'';
then, The Matrix Reloaded \cite{wachowski03reloaded} used a technique called Universal Capture by Borshukov et al. \cite{borshukov05universal} to capture the skin surface of a human head, in order to replicate the head in the form of computer models.

Recent movies and video games (such as \cite{rockstar2011noire}) have shown that by capturing real motion of facial expressions instead of simulating them produces impressive results.
Literature on capture of human faces and skin surface motions is large. \cite{deng2007computer}
Professional quality capture still needs a laboratory environment with a large number of cameras \cite{winder2008technical,motionscan}.

The movie industry usually knows the structure of the objects that are tracked; a pre-recorded mesh is used as a helping model (also called ``virtual bones'' to track the object pose. \cite{todo} % universal capture
It is obviously easier to map image features to a priori information than to recover a fully unknown structure.
(A pre-recorded higher resolution texture could be used for pore-level details, then tracking with a good-enough resolution and possibly markers to get wrinkles.)

% Realtime capture can be sped up by modeling the face in a parametric space, and mapping video frames to the model.

The reconstruction and rendering involves lots of filtering of the raw data and further using high-resolution textures to extract better resolution depth data than what is available with computer vision technologies only.

% XXX less background, more contents and motivation

Motivation: such uses, very popular, much algorithms, need own scanner wow.

\subsection{Goal and scope of the thesis}

% should the face stuff be moved under here? probably parts of it?

% choices {

The objective of this work is to describe a complete 3D object scanning and reconstruction pipeline for both shape and texture information.
A digital camera based reconstruction rig is constructed, supported by software tools for acquiring both static image data and video files.
Feasible readily available programs for reconstructing 3D geometry and texture based on the captured image data is surveyed and presented, for producing textured 3D meshes and/or point clouds for further handling.

Real-time processing is not considered important, as the data will be post-processed and observed manually for maximum quality.

or

The intention of this work is both capturing the structure and texture of a target and tracking of the target's movement by time.
The captured data should then be used to reconstruct the same target.
A camera rig is constructed to record continuous video from several, carefully decided viewpoints.
In addition to multiple cameras at different locations and poses, an uniform lightning is also required to minimize specular difficulties in the texture.

or

The aim of this work is to evaluate both software and hardware aspects in three-dimensional object reconstruction, and to develop such system, targeted to small sized objects.
Both the geometric properties and color texture were considered.
Aim is in hardware robustness, ease of use and general extendability. While there are no detailed plans on further use of the machinery, it should support many kinds of studies in the future of its structure and without knowing all implementation issues in detail.

A special case subject for scanning in this work is the human face: an interesting target because of its complex surface material and diverse ability to produce many different expressions.
Many studies have recently focused on facial 3D reconstruction based on a set of high resolution photographs.

% }

Basic techniques for taking also motion into account are presented, but the use of state-of-the-art algorithms is left for future implementations.

% "a rig is built" or "a rig was built"?

Constructing a rig means studying the theoretical background for the reconstruction process in order to be familiar with the system needs, selecting hardware and setting it up, and writing software for using the selected cameras to automate the scanning process.
A reconstruction rig consisting of digital cameras can be fixed in an indoor setup, or a more flexible setting that can be transported to a location where it is needed.
In this work, transportability and flexibility were considered important, and the final product was successful in that it can be packed and transported, set up, and used in a relatively simple manner.
The rig will be used for future research on image-based reconstruction algorithms.


The thesis is organised as follows.
Theoretical background on the elementary methods used in 3D reconstruction implementations are described in sections A-B, focusing on image acquisition, multi-view 3D reconstruction and 3D motion capture, respectively.
In section C, the practical design and implementation of the aim of the thesis, the flexible 3D scanning rig, is presented.
Section D focuses on using the rig for case experiments and testing it on some research questions.
Feasibility of the rig, possible uses and future work are discussed in chapter E.
Chapter F gives a conclusion on the work.
