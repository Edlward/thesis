%Tutkimusaineisto ja -menetelmät
\section{Materials and methods}

\subsection{Available software tools}

\subsubsection{Programs}

- kinect
- autodesk 123d catch
- meshlab
- meshmixer
- photofly
- sfm toolkit

\subsubsection{Libraries}

- OpenCV
- PCL
- KLT: http://www.inf.ethz.ch/personal/chzach/opensource.html
- http://slowmovideo.granjow.net/


\subsection{Hardware construction}

\subsubsection{Frame}

Aluminium profile system as a frame

Generic camera support screws, 360 angle ball joints?

Adapters for machine vision cameras

Arduino-like adapter HW for sync signals

Connectors, wire

\subsubsection{Cameras}

GigE/USB3/???

- size
- resolution
- speed
- cmos/ccd
- configurability
- noise
- price
- availability

\subsection{Camera matrix}

not considered: weatherproofing, lcd/viewfinder/user interface, mirror blackout, autofocus, etc

resolution
dynamic range
burst mode speed
wire remote shutter speed
dust reduction
optical stabilizing
video frame rate, resolution
usb speed
usb mass storage
flash something
manual mode settings
shutter lag
mirror lockup
weight
price

\subsection{Practicalities}
% or \subsection{Data recording}

- lens distortion?
- rigid base, motion blur
- baseline width, focus, depth, fstop etc
- compression artifacts are nasty (edge detectors go wild etc.)


\subsection{Reconstruction}

Both shape and texture are considered in this work. Only diffuse color (albedo) is of interest; more complex material properties are assumed to be captured in other means and not spatially varying.

Basic uv mapping. Project texture to computed mesh. Somehow use colors and optical flow everywhere...

Postprocessing: remodel the mesh (face), see what it would look like. Refine parameters to get a similar output as in the photos (normal map etc.), backproject. Use colors and highpass them; assume uniform lightning and locally uniform texture color (bradley). (Simply a rendering technique, that level of detail in 3D structure might not be needed).


KINECT 2 HYBRID SHIT YEA
