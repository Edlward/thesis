%Yhteenveto
% "further work remains to .."
\section{Conclusion} \label{sec:conclusion} % conclusion/conclusions depending on the length

Realistic content is increasingly needed in computer graphics.
Current processing power of desktop computers with graphics processing units can utilize millions of pixels of texture data in 3D models with ease, and the available power is still increasing.

Computer graphics has a clear connection to computer vision.
Areas of computer graphics focus more on synthesizing content on a computer's screen, while computer vision analyses images from real life.
However, they do use similar principles, and vision can be used to create model data for rendered graphics.

Scanning real-life objects directly brings TODO TODO.

In this thesis, a multi-view stereo rig was constructed with nine consumer cameras for scanning small subjects, such as human faces.

Requirements for technical details were specified, based on higher-level specifications on the system usage.
At the core of the design is nine Canon EOS 700D digital cameras with APS-C sensor sizes, utilizing 50 mm full-frame equivalent lenses, resulting in 18-megapixel images and a total rig footprint of a few squaremetres for most subjects.
Auxiliary hardware and software were built to aid data synchronization and acquisition, making the photo scanning process nearly automatic.
With proper post-processing, the system is feasible for scanning both static high-resolution subjects, and dynamic subjects with less spatial resolution.

Synchronization of still and video recording were evaluated. TODO.
% Scanning accuracy TODO


% (summary)
